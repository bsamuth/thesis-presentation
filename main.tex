\documentclass[aspectratio=169, 22pt]{beamer}
\makeatletter
\def\input@path{{./sty/}}
\makeatother

\usetheme{greyc}
%Choose your language
\def\lang{french} %set it to english or french to have the logo in the good language

\usepackage{amsmath,amsfonts,amssymb,latexsym}
\usepackage{mathtools,bbm}
\usepackage[justification=centering]{caption}
\usepackage[justification=centering]{subcaption}

\graphicspath{{./img/},{./figs/}}

\captionsetup[figure]{labelformat=empty,justification=centering}

\DeclareMathOperator*{\argmin}{argmin}
\def\x{{\mathbf x}}
\def\L{{\cal L}}
\DeclarePairedDelimiter\ceil{\lceil}{\rceil}
\DeclarePairedDelimiter\floor{\lfloor}{\rfloor}



\title[Modèles hybrides pour la génération d'images]{Des modèles hybrides combinant réseaux de neurones et méthodes à patchs pour la génération d'images}
\subtitle{}
\author[Benjamin Samuth]{Benjamin Samuth}
\institute[Normandie University]{Normandie Univ, UNICAEN, ENSICAEN, CNRS, GREYC, Caen, FRANCE}
\email{benjamin.samuth@unicaen.fr}
\web{}

\begin{document}

\begin{frame}
  \titlepage
\end{frame}

% ============================================
% ============================================
\section{Introduction}

\subsection{Contexte}
\begin{frame}{\secname~- \subsecname}
  \framesubtitle{L'évolution de la Génération d'images}
      \begin{itemize}
      \item Vaste ensemble d'applications
      \item Principalement à usage artistique
      \item Apprentissage profond $\Rightarrow$ \alert{Nouveau paradigme}
      \end{itemize}
      
      \begin{figure}
        \captionsetup[subfigure]{labelformat=empty}
        \centering
        \begin{subfigure}[t]{0.19\linewidth}\centering
          \includegraphics[width=\linewidth]{texture.png}
          \caption{{\scriptsize Synthèse de texture \newline {\tiny [Ashikhmin et al., 2001]}}}
        \end{subfigure}
        \begin{subfigure}[t]{0.19\linewidth}\centering
          \includegraphics[width=\linewidth]{patchmatch.png}
          \caption{{\footnotesize Édition \newline {\scriptsize [Barnes et al., 2009]}}}
        \end{subfigure}        
        \begin{subfigure}[t]{0.19\linewidth}\centering
          \includegraphics[width=\linewidth]{gatys}
          \caption{{\footnotesize Transfert de style \newline {\scriptsize [Gatys et al., 2014]}}}
        \end{subfigure}
        \begin{subfigure}[t]{0.19\linewidth}\centering
          \includegraphics[width=\linewidth]{stylegan}
          \caption{{\footnotesize Génération de visages (\emph{StyleGAN}) \newline {\scriptsize [Karras et al., 2019]}}}
        \end{subfigure}        
        \begin{subfigure}[t]{0.19\linewidth}\centering
          \includegraphics[width=\linewidth]{stable-diffusion}
          \caption{{\footnotesize \emph{Text-to-image} (Diffusion) \newline {\tiny [Rombach et al., 2022]}}}
        \end{subfigure}
      \end{figure}
\end{frame}
% Fait pendant la thèse

\begin{frame}{\secname~- \subsecname}
  \framesubtitle{Le prix de la performance}
  \begin{figure}
    \begin{subfigure}[t]{0.24\linewidth}\centering
      \includegraphics[width=\linewidth]{gatys}
      \caption{{\emph{Neural style transfer}\newline[Gatys et al., 2014]\newline 3M de paramètres}}
    \end{subfigure}
    \begin{subfigure}[t]{0.24\linewidth}\centering
      \includegraphics[width=\linewidth]{stylegan}                
      \caption{{\emph{StyleGAN}\newline[Karras et al., 2019]\newline 50M de paramètres}}
    \end{subfigure}
    \begin{subfigure}[t]{0.24\linewidth}\centering
      \includegraphics[width=\linewidth]{esser}                      
      \caption{\emph{Taming transformers}\newline[Esser et al., 2021]\newline 800M de paramètres}
    \end{subfigure}
    \begin{subfigure}[t]{0.24\linewidth}\centering
      \includegraphics[width=\linewidth]{stable-diffusion}
      \caption{\emph{Stable Diffusion}\newline[Rombach et al., 2022]\newline 1Mrd de paramètres}
    \end{subfigure}
  \end{figure}
  \begin{exampleblock}{}
    \centering
    + de qualité, + de capacité $\Rightarrow$ \textbf{+ de \alert{paramètres}, + de \alert{données}}  d'entraînement
  \end{exampleblock}
\end{frame}


\subsection{Enjeux}
\begin{frame}{\secname~- \subsecname}
  \framesubtitle{Enjeux}
  \begin{columns}
    \begin{column}{0.5\linewidth}
      \begin{customblock}{Enjeux scientifiques}
        \begin{itemize}
        \item Contraintes matérielles
        \item Reproductibilité
        \end{itemize}
      \end{customblock}
      \pause
      \begin{block}{Enjeux éthiques}
        \begin{itemize}
        \item Sécurité des données
        \item Violation des droits d'auteurs
        \end{itemize}
      \end{block}
    \end{column}
    \begin{column}{0.5\linewidth}
      \pause
      \begin{alertblock}{Enjeux de société}
        \begin{itemize}
        \item Désinformation
        \item Propagande de guerre % Enlever
        \end{itemize}
      \end{alertblock}
      \pause
      \begin{exampleblock}{Enjeux environementaux}
        \begin{itemize}
        \item Coût colossal en énergie (fermes de GPUs, entraînement + utilisation)
        \end{itemize}
      \end{exampleblock}
    \end{column}
  \end{columns}
\end{frame}

\subsection{Modèles légers}
% \begin{frame}{\secname~- \subsecname}
%   \begin{exampleblock}{\centering Quels critères pour faire face à ces enjeux ?}
%     \begin{itemize}
%       \pause
%     \item Des modèles \alert{légers}
%       \pause
%     \item Des modèles \alert{explicables}
%     \end{itemize}
%     \pause
%     ... mais tout de même les plus performants possibles~!
%   \end{exampleblock}
% \end{frame}

\begin{frame}{\secname~- \subsecname}
  \framesubtitle{Objectifs}
  \begin{columns}
    \begin{column}{0.5\linewidth}
      \begin{customblock}{Modèles légers / frugaux}
        \begin{itemize}
        \item \small Paramètres
        \item \small Données d'entraînement
        \item \small Opérations possible
        \end{itemize}
      \end{customblock}
      \begin{alertblock}{Challenges}
        \begin{itemize}
        \item \small Compromis qualité/diversité
        \item \small Champs d'action limité (ex: 1 seule image)
        \end{itemize}
      \end{alertblock}
    \end{column}
    \begin{column}{0.5\linewidth}
      \begin{figure}
        \centering
        \begin{subfigure}{0.45\linewidth}
          \centering
          \includegraphics[width=0.8\linewidth]{fastgan.png}
          \caption{Génération \emph{few-shot} (FastGAN) [Liu et al., 2020]}
        \end{subfigure}
        
        \begin{subfigure}{\linewidth}
          \centering
          \includegraphics[width=0.8\linewidth]{granot.png}
          \caption{\emph{GPNN} (\emph{``Drop the GAN''}) [Granot et al., 2022]}
        \end{subfigure}
      \end{figure}
    \end{column}
  \end{columns}
\end{frame}
% Explicable à posteriori
% Explicable par design ?

\subsection{Modèles explicables}
\begin{frame}{\secname~- \subsecname}
  \framesubtitle{Objectifs}
  \begin{columns}
    \begin{column}{0.5\linewidth}
      \begin{block}{Modèles explicables}        
        \begin{itemize}
        \item \small Sujet émergent
        \item \small Comprendre comment et pourquoi 
        \item \small Connaître les capacités et limites
        \item \small Offrir des garanties (biais, sécurité des données, ...)
        \end{itemize}
      \end{block}
      \begin{alertblock}{Limitations et Challenges}
        \begin{itemize}
        \item \small \emph{A posteriori}
        \item \small Inexistant pour les modèles génératifs
        \end{itemize}
      \end{alertblock}
    \end{column}
    \begin{column}{0.5\linewidth}
      \begin{figure}
        \centering
        \includegraphics[width=1\linewidth]{sauer}
        \caption{{Explications contrefactuelles \newline [Sauer et al., 2021]}}
      \end{figure}
    \end{column}
  \end{columns}    
\end{frame}

\begin{frame}{\secname~- \subsecname}
  \begin{customblock}{\centering Problématique}
    \centering
    Peut-on concilier la qualité de génération (photoréalisme) avec
    les critères de légereté et d'explicabilité des modèles ?
  \end{customblock}
\end{frame}

\subsection{Méthodes ``hybrides''}
\begin{frame}{\secname~- \subsecname}
  \framesubtitle{Motivation} 
  \begin{columns}
    \begin{column}{0.5\linewidth}
      \begin{block}{Modèles profonds}
        \begin{itemize}
        \item \small Photoréalistes (qualité)
        \item \small Génération variée (diversité)
        \item \small Non-explicables et complexes
        \end{itemize}
      \end{block}      
    \end{column}

    \pause
    \begin{column}{0.5\linewidth}
      \begin{customblock}{Méthodes frugales}
        \begin{itemize}
        \item \small Légères, explicables
        \item \small Compromis qualité/diversité
        \item \small Champ d'applications restreint
        \end{itemize}
      \end{customblock}      
    \end{column}    
  \end{columns}

  \vfill
  \pause
  \begin{exampleblock}{\centering Des méthodes hybrides pour la génération d'images}
    \centering
    \alert{Idée}~: Combiner méthodes frugales et modèles profonds pour
    conserver les propriétés pertinentes de chacun
  \end{exampleblock}
  \begin{itemize}
    \centering
    \item Une piste encore peu explorée...!
  \end{itemize}
\end{frame}

\subsection{Méthodes ``hybrides''}
\begin{frame}{\secname~- \subsecname}
  \framesubtitle{Motivation}
  \begin{columns}
    \begin{column}{0.3\linewidth}
      \begin{itemize}
        \centering
        \item \small 1. Exposer \alert{qualitativement} leur pertinence dans le transfert de style
      \end{itemize}
    \end{column}
    \begin{column}{0.3\linewidth}
      \begin{itemize}
        \centering
      \item \small 2. Démontrer leurs \alert{économies en calculs} dans la génération \emph{few-shot} de visages
      \end{itemize}      
    \end{column}
    \begin{column}{0.3\linewidth}
      \begin{itemize}
        \centering
      \item \small 3. Proposer une voie pour les \alert{généraliser} sur plus de données variées
      \end{itemize}      
    \end{column}
  \end{columns}
\end{frame}


% ============================================
% ============================================
\section{Concepts préliminaires}
% Justifier leur utilisation
\begin{frame}
  \begin{beamercolorbox}[sep=15pt,center,shadow=true,rounded=true]{title}
    \LARGE\bfseries \secname
  \end{beamercolorbox}
\end{frame}

\subsection{Encodeurs}
\begin{frame}{\secname~- \subsecname}
  \framesubtitle{Principe}
  \begin{columns}
    \begin{column}{0.4\linewidth}
      \begin{block}{Encodeurs}
        \begin{itemize}
        \item Entraînement supervisé (classification)
        \item Usage : features, métriques
        \end{itemize}
      \end{block}
    \end{column}
    \begin{column}{0.6\linewidth}
      \begin{figure}
        \centering
        \includegraphics[height=0.6\textheight]{vgg16.png}
        \caption{\emph{VGG-16} [Simonyan et al., 2014]}
      \end{figure}
    \end{column}
  \end{columns}
\end{frame}

\subsection{Auto-encodeurs}
\begin{frame}{\secname~- \subsecname}
  \begin{figure}
    \centering
    \includegraphics[width=0.7\linewidth]{4.png}
    \caption{Schéma classique d'un auto-encodeur}
    \begin{block}{Reconstruction}
      \centering
      \begin{itemize}        
      \item Optimisation d'un objectif de reconstruction
      \end{itemize}
      $\underset{E,D}{\min}\ \underset{x \sim p(x)}{\mathbb{E}}\left[\mathcal{L}_{\text{rec}}(x,D(E(x)))\right]$
      \begin{itemize}
      \item Représentation \alert{latente} et compacte
      \end{itemize}

    \end{block}
  \end{figure}
  
\end{frame}

\subsection{Méthodes à patchs}
% Alléger déf
% parler de patchmatch
\begin{frame}{\secname~- \subsecname}
  \framesubtitle{Définition}
  \begin{columns}
    \begin{column}{0.45\linewidth}
      \begin{figure}
        \centering
        \includegraphics[width=\linewidth]{2.png}
      \end{figure}
    \end{column}
    
    \begin{column}{0.55\linewidth}
      % \emph{Image} : $x \in \mathbb{R}^{W \times H \times C}$
      % \begin{itemize}
      %   \footnotesize 
      % \item de taille $W \times H$
      % \item à $C$ canaux de couleurs
      % \end{itemize}
      
      % \vfill
      % \emph{Extraction de patch} : $P$
      % \begin{itemize}
      %   \footnotesize 
      % \item Opérateur linéaire
      % \item Extrait des patchs de taille $\sigma \times \sigma$
      % \item Gère les effets de bords
      % \end{itemize}

      % \vfill
      % \emph{Position} : $i \in \Omega$
      % \begin{itemize}
      %   \footnotesize 
      % \item $\Omega = \{1,\dots,W\}\times\{1,\dots,H\}$
      % \end{itemize}

      % \vfill
      % \emph{Patch} : $P(x,i) \in \mathbb{R}^{\sigma \times \sigma \times C}$
      % \begin{itemize}
      %   \footnotesize 
      %   \item $\forall (u,v) \in \{0,\dots,\sigma-1\}^2,\ P(x,i)_{u, v, \cdot} =
      %     x_{i+(u,v)}$
      %   \item par simplicité, sous forme vectorisée (\emph{ie.} $\mathbb{R}^{\sigma^2 C}$)
      % \end{itemize}     
    \end{column}
  \end{columns}  
\end{frame}

% Warping
% Méthode de génération pour les méthodes à patchs
% Notation R o W
\begin{frame}{\secname~- \subsecname} 
  \framesubtitle{Warping}
  \begin{columns}
    \begin{column}{0.6\linewidth}
      \begin{exampleblock}{Carte d'assignement de patchs}
        \begin{itemize}
        \item Permet l'agrégation/le \emph{warping} de patchs
        \item (Re)construire une image à partir d'un \alert{assignement de coordonnées}
        \item Génération explicable
        \end{itemize}
      \end{exampleblock}
    \end{column}
    \begin{column}{0.4\linewidth}
      \begin{figure}
        \centering
        \includegraphics[height=0.6\textheight]{nnf.png}
        \caption{(a) Image $x$ (b) Identité \\ (d) Image $y$ (c) Carte d'assignement}
      \end{figure}
    \end{column}
  \end{columns}
\end{frame}


% ============================================
% ============================================
\section{Transfert de style par contrainte de patchs}
\begin{frame}
  \begin{beamercolorbox}[sep=15pt,center,shadow=true,rounded=true]{title}
    \LARGE\bfseries Contribution 1.\\ \secname    
  \end{beamercolorbox}
  TODO: Publication ICIP GRETSI
\end{frame}

\subsection{Contexte}
% Image I, R, noter J
\begin{frame}{\secname~- \subsecname}
  \begin{columns}
    \begin{column}{0.6\linewidth}
      \begin{customblock}{Principe du transfert de style}
        \begin{itemize}
        \item \small Style : ensemble de \alert{caractéristiques esthétiques} propres à une image
        \item \small Transfert de style : poser le style artistique d'une
          image de \alert{référence} $R$ sur une image de \alert{contenu} $I$ (\emph{Input})
        \end{itemize}    
      \end{customblock}
      \begin{block}{\emph{Neural style transfer} [L. Gatys, 2016]}
        \begin{itemize}
        \item \small Optimise les pixels d'une image selon les
          \emph{features} \emph{VGG-16} [Simonyan et al., 2014]
        \item \small Procédé \alert{long} (5-15 min. sur GPU par image)
        \item \small À répéter pour chaque différent style/contenu
        \end{itemize}
      \end{block}
    \end{column}
    \begin{column}{0.4\linewidth}
      \begin{figure}
        \centering
        \includegraphics[width=\linewidth]{balloon/gatys.png}
        \caption{Méthode de [L. Gatys, 2016]}
      \end{figure}
    \end{column}
  \end{columns}
\end{frame}

\subsection{Méthode}
\begin{frame}{\secname~- \subsecname}
  \begin{figure}
    \centering
    \includegraphics[width=\linewidth]{5.pdf}
    \caption{Schéma du transfert de style à une échelle $k$.}
  \end{figure}
\end{frame}

\begin{frame}[noframenumbering]{\secname~- \subsecname}
  \begin{figure}
    \centering
    \includegraphics[width=\linewidth]{5_1}
    \caption{Schéma du transfert de style à une échelle $k$.}
  \end{figure}
\end{frame}

\begin{frame}[noframenumbering]{\secname~- \subsecname}
  \begin{figure}
    \centering
    \includegraphics[width=\linewidth]{5_2}
    \caption{Schéma du transfert de style à une échelle $k$.}
  \end{figure}
\end{frame}

\begin{frame}[noframenumbering]{\secname~- \subsecname}
  \begin{figure}
    \centering
    \includegraphics[width=\linewidth]{5_3}
    \caption{Schéma du transfert de style à une échelle $k$.}
  \end{figure}
\end{frame}

\begin{frame}[noframenumbering]{\secname~- \subsecname}
  \begin{figure}
    \centering
    \includegraphics[width=\linewidth]{5_4}
    \caption{Schéma du transfert de style à une échelle $k$.}
  \end{figure}
\end{frame}

\begin{frame}{\secname~- \subsecname}
  \framesubtitle{Multi-échelle}
  \begin{columns}
    \begin{column}{0.5\linewidth}
      \begin{itemize}
      \item $W_k$~: carte des patchs plus proches voisins de $J_k$ dans $R_k$.
      \item $W_k$ contient des \alert{coordonnées}
      \item Passage d'échelle d'un ratio $r > 1$
      \end{itemize}
      \begin{figure}
        \includegraphics[width=0.8\linewidth]{5_me.png}
      \end{figure}

    \end{column}
    \begin{column}{0.5\linewidth}
      \begin{block}{Passage à l'échelle}
        \centering
        $W\!\!\uparrow^r(p) = \floor*{rW\left(\floor*{\frac{p}{r}}\right) + p - r\floor*{\frac{p}{r}}}$
      \end{block}
      \vspace{1em}
      \begin{exampleblock}{Avantages}
        \begin{itemize}
        \item Opération immédiate
        \item Pas de perte de qualité
        \end{itemize}
      \end{exampleblock}
    \end{column}
  \end{columns}
\end{frame}

\begin{frame}{\secname~- \subsecname}
  \framesubtitle{Pénalisation d'occurrence}
  \begin{columns}
    \begin{column}{0.5\linewidth}
        \begin{itemize}
        \item Recherche de + proches voisins \emph{``trop''} efficaces
        \item Équilibrer l'utilisation des patchs de style
        \end{itemize}

      \begin{block}{Distance de patchs pénalisée}
        \centering
        \small
        $W_k^* \in \argmin_{W \in F}\ \sum_{p \in \Omega} \frac{D_k(p)}{Var(D_k)} - \lambda_W \circ W(p)$
        
        $\text{avec} \quad D_k(p) = ||\tilde{J}_k(p) - R_k \circ W(p)||^2$
      \end{block}
      \begin{itemize}
      \item $\lambda_W \circ W(p) < 0$~: Patch en $p$ pénalisé
      \item $\lambda_W \circ W(p) > 0$~: Patch en $p$ favorisé
      \end{itemize}
    \end{column}
    \begin{column}{0.5\linewidth}
      \begin{figure}
        \centering
        \includegraphics[width=\linewidth]{6.png}
      \end{figure}
    \end{column}
  \end{columns}
\end{frame}

\begin{frame}{\secname~- \subsecname}
  \framesubtitle{Pénalisation d'occurrence}
  \begin{columns}
    \begin{column}{0.5\linewidth}      
      \begin{block}{Actualisation de la pénalisation}
        \small
        \begin{center}
          $\lambda_W(y) \leftarrow \lambda_W(y) + \delta \partial_y \lambda_W(y) $
          
          $\partial_y\lambda_W(y) = \frac{\lvert\left\{p\ |\ y =
              W(p)\right\}\rvert}{\lvert\Omega_I\rvert} -
          \frac{\nu(y)}{\lvert\Omega_R\rvert}$
        \end{center}
        
        \begin{itemize}
        \item Méthode inspirée par le transport optimal
        \item Durant les différentes itérations de \emph{PatchMatch} [C.~Barnes, 2009]
        \item Converge vers un point fixe
        \end{itemize}
      \end{block}    
    \end{column}
    
    \begin{column}{0.5\linewidth}
      \begin{figure}
        \centering
        \includegraphics[width=\linewidth]{6.png}
      \end{figure}
    \end{column}
  \end{columns}
\end{frame}

\subsection{Résultats}
\begin{frame}{\secname~- \subsecname}
  \begin{figure}
    \begin{subfigure}{0.47\linewidth}
      \includegraphics[width=\linewidth]{balloon/ours.png}
      \caption{Notre méthode ($\Phi$ : RGB \& Gradients)}
    \end{subfigure}
    \begin{subfigure}{0.45\linewidth}
      \begin{subfigure}{0.45\linewidth}
        \includegraphics[width=\linewidth]{balloon/balloon_s.png}
        \caption{Contenu}
      \end{subfigure}
      \begin{subfigure}{0.45\linewidth}
        \includegraphics[width=\linewidth]{balloon/starry_night_s.jpg}
        \caption{Style}
      \end{subfigure}

      \begin{subfigure}{0.45\linewidth}
        \includegraphics[width=\linewidth]{balloon/gatys.png}
        \caption{[Gatys et al., 2014]}
      \end{subfigure}
      \begin{subfigure}{0.45\linewidth}
        \includegraphics[width=\linewidth]{balloon/liu.png}
        \caption{[Liu et al., 2021]}
      \end{subfigure}
    \end{subfigure}
  \end{figure}
\end{frame}

\begin{frame}{\secname~- \subsecname}
  \begin{figure}
    \begin{subfigure}{0.47\linewidth}
      \includegraphics[width=\linewidth]{tuebingen/ours.png}
      \caption{Notre méthode ($\Phi$ : \emph{VGG-16}, relu\_1\_1)}
    \end{subfigure}
    \begin{subfigure}{0.45\linewidth}
      \begin{subfigure}{0.45\linewidth}
        \includegraphics[width=\linewidth]{tuebingen/tuebingen_s.jpg}
        \caption{Contenu}
      \end{subfigure}
      \begin{subfigure}{0.45\linewidth}
        \includegraphics[width=\linewidth]{tuebingen/abstract_s}
        \caption{Style}
      \end{subfigure}

      \begin{subfigure}{0.45\linewidth}
        \includegraphics[width=\linewidth]{tuebingen/gatys.png}
        \caption{[Gatys et al., 2014]}
      \end{subfigure}
      \begin{subfigure}{0.45\linewidth}
        \includegraphics[width=\linewidth]{tuebingen/liu.png}
        \caption{[Liu et al., 2021]}
      \end{subfigure}
    \end{subfigure}
  \end{figure}
\end{frame}


\begin{frame}{\secname~- \subsecname}
  \begin{columns}
    \begin{column}{0.5\linewidth}
      \begin{figure}
        \centering
        \includegraphics[width=0.9\linewidth]{7.png}
      \end{figure}
    \end{column}
    \begin{column}{0.5\linewidth}
      \begin{block}{Pertinence des métriques}
        \begin{itemize}
        \item \emph{``Patch-features''} $\Phi$
        \item Mise en évidence : stylisation d'une image stylisée
        \item Caractéristiques stylistiques capturées différentes selon $\Phi$
        \end{itemize}
      \end{block}
    \end{column}
  \end{columns}
\end{frame}


% ============================================
% ============================================
\section{Patchs latents pour la génération}
\begin{frame}
  \begin{beamercolorbox}[sep=15pt,center,shadow=true,rounded=true]{title}
    \LARGE\bfseries Contribution 2.\\ \secname
  \end{beamercolorbox}
\end{frame}

\subsection{Contexte}
% \begin{frame}{\secname~- \subsecname}
%   \framesubtitle{Génération \emph{``few-shot''}}
%   \begin{itemize}
%   \item Génération avec peu de données
%   \item Entre 10 et 1000 échantillons
%   \item Problème difficile : instabilité ou mémorisation
%   \end{itemize}
%   \begin{customblock}{}
%     \centering
%     Certaines méthodes de l'état de l'art (\emph{transformers},
%     diffusion) sont \alert{inapplicables} telles quelles dans ces
%     configurations !
%   \end{customblock}
%   \pause
%   \begin{itemize}
%   \item Solutions : Architectures, augmentation de données, transfert de connaissances, ...
%   \item \alert{Inconvénient} : phase d'apprentissage longue malgré le peu de données ($\sim$2h)
%   \end{itemize}  
% \end{frame}

\subsection{\emph{VQGAN}}
\begin{frame}{\secname~- \subsecname}
  \framesubtitle{[Esser et al., 2021]}
  \begin{columns}
    \begin{column}{0.2\linewidth}
      \begin{itemize}
      \item \small Auto-encodeur
      \item \small Adversaire
      \item \small Quantifié
      \item \small Spatial
        \vspace{1em}
      \item \small Génération par un \emph{Transformer}
      \end{itemize}
    \end{column}
    \begin{column}{0.8\linewidth}
      \begin{figure}
        \includegraphics[width=\linewidth]{10.png}
      \end{figure}
    \end{column}
  \end{columns}
\end{frame}

\begin{frame}{\secname~- \subsecname}
  \framesubtitle{Propriété}
  \begin{figure}
    \includegraphics[width=\linewidth]{16.png}
  \end{figure}
  \begin{itemize}
  \item \alert{Estompe les bords} de manière photoréaliste
  \item \alert{Harmonise} les couleurs/textures du visage.
\end{itemize}
\end{frame}

\begin{frame}{\secname~- \subsecname}
  \framesubtitle{Pertinence de l'auto-encodeur}
  \begin{figure}
    \includegraphics[width=\linewidth]{15.png}
  \end{figure}
  \begin{itemize}
  \item L'auto-encodeur \alert{n'est pas} un générateur à lui seul.        
  \end{itemize}
\end{frame}


\subsection{Méthode}
\begin{frame}{\secname~- \subsecname}
  \begin{figure}
    \includegraphics[width=\linewidth]{11.pdf}
  \end{figure}
  \begin{itemize}
  \item $E, D$ : VQGAN pré-entraîné sur FFHQ
  \item $\mathcal{X}$ : \emph{Dataset} limité
  \item $R$ : Réduction de dim. par ACP
  \item $G$ : Génération d'une représentation latente par patchs
  \end{itemize}
\end{frame}

\begin{frame}{\secname~- \subsecname}
  \framesubtitle{Génération séquentielle auto-régressive}
  \begin{figure}
    \includegraphics[width=0.8\linewidth]{12.pdf}
  \end{figure}
\end{frame}

\subsection{Expériences}
\begin{frame}{\secname~- \subsecname}
  \framesubtitle{Résultats}
  \begin{columns}
    \begin{column}{0.5\linewidth}
      \begin{figure}
        \includegraphics[width=\linewidth]{13a}
        \includegraphics[width=\linewidth]{13b}
      \end{figure}
    \end{column}
    \begin{column}{0.5\linewidth}
      \begin{figure}
        \includegraphics[width=\linewidth]{14a}
        \includegraphics[width=\linewidth]{14b}
      \end{figure}
    \end{column}
  \end{columns}
\end{frame}

\begin{frame}{\secname~- \subsecname}
  \framesubtitle{Explicable par design}
  \begin{columns}
    \begin{column}{0.4\linewidth}
      \begin{itemize}
      \item Relation \alert{source} et \alert{génération} immédiate
      \end{itemize}
    \end{column}
    \begin{column}{0.6\linewidth}
      \begin{figure}
        \includegraphics[width=\linewidth]{17.png}
      \end{figure}  
    \end{column}
  \end{columns}  
\end{frame}

\subsection{Évaluation}
\begin{frame}{\secname~- \subsecname}
  \framesubtitle{Comparaison qualitative}
  \begin{columns}
    \begin{column}{0.75\linewidth}
      \begin{figure}
        \includegraphics[width=\linewidth]{18.png}
      \end{figure}
    \end{column}
    \begin{column}{0.25\linewidth}
      \begin{itemize}
      \item Comparaison avec \emph{FastGAN}
      \item \emph{LatentPatch} est bien + varié
      \end{itemize}
    \end{column}
  \end{columns}  
\end{frame}

\begin{frame}{\secname~- \subsecname}
  \framesubtitle{Comparaison quantitative - FID}
  \begin{figure}
    \includegraphics[width=0.8\linewidth]{fid-vs-subset.pdf}
  \end{figure}
\end{frame}

\subsection{Généralisation}
\begin{frame}{\secname~- \subsecname}
  \framesubtitle{Tentative de généralisation}
  \begin{columns}
    \begin{column}{0.6\linewidth}
      \begin{figure}
        \centering
        \includegraphics[width=\linewidth]{latentpatch_full.png}
      \end{figure}
    \end{column}
    \begin{column}{0.4\linewidth}
      \begin{exampleblock}{\emph{LatentPatch}}
        \begin{itemize}
        \item Parfait en \emph{few-shot} : visages perceptuellement proches
        \end{itemize}
      \end{exampleblock}
      \begin{alertblock}{\emph{LatentPatch} généralisé}
        \begin{itemize}
        \item Sensible aux échantillons qui varient
        \item Manque de cohérence spatiale
        \end{itemize}
      \end{alertblock}
    \end{column}
  \end{columns}
\end{frame}

% ============================================
% ============================================
\section{Modèle de mélanges gaussiens latents}
% INtroduire le problème
% Cohérence spatiale
% Modèle gaussien pour capturrer

% Motiver par faible rang sampling
\begin{frame}
  \begin{beamercolorbox}[sep=15pt,center,shadow=true,rounded=true]{title}
    \LARGE\bfseries Contribution 3. \\ \secname
  \end{beamercolorbox}
\end{frame}

\subsection{Méthode}
% Idée: Contraindre les corrélations spatiales en exploitant les covariances des données
% Illustrer bruit blanc vs bruit corrélé
\begin{frame}{\secname~- \subsecname}
  \begin{figure}
    \includegraphics[width=0.9\linewidth]{lrgmm.pdf}
  \end{figure}
  \begin{columns}
    \begin{column}{0.5\linewidth}
      \begin{exampleblock}{Principe}
        \begin{itemize}
        \item \small Apprendre un \emph{GMM} sur les représentations latentes
        \item \small Échantillonner le \emph{GMM} à un faible rang comme initialisation
        \end{itemize}
      \end{exampleblock}
    \end{column}
    \begin{column}{0.5\linewidth}
        \begin{exampleblock}{}
        \begin{itemize}
        \item \small Projeter chaque patch au plus proche voisin
        \item \small Décoder
        \end{itemize}
      \end{exampleblock}
    \end{column}
  \end{columns}
  
\end{frame}

\begin{frame}{\secname~- \subsecname}
  \framesubtitle{Échantillonnage à faible rang} 
  \begin{figure}
    \includegraphics[width=0.9\linewidth]{ablation_dk.png}
  \end{figure}  
\end{frame}

\begin{frame}{\secname~- \subsecname}
  \framesubtitle{Raffinement par recherche de patchs plus proches voisins}
  \begin{figure}
    \includegraphics[width=0.9\linewidth]{ablation_nn.png}
  \end{figure}    
\end{frame}

\subsection{Résultats}
\begin{frame}{\secname~- \subsecname}
  \framesubtitle{Comparaisons qualitative}
  \begin{figure}
    \includegraphics[width=\linewidth]{comp_img}
  \end{figure}
\end{frame}

% ============================================
% ============================================
\section{Conclusion et Perspectives}
\begin{frame}
  \begin{beamercolorbox}[sep=15pt,center,shadow=true,rounded=true]{title}
    \LARGE\bfseries \secname
  \end{beamercolorbox}
\end{frame}

\subsection{Résumé}
\begin{frame}{\secname~- \subsecname}
  \begin{customblock}{Qualité, légèreté, explicabilité}
    \begin{itemize}
    \item Problème : la croissance des modèles 
    \item La légèreté et l'explicabilité sont souhaitables
    \item Les modèles hybrides peuvent répondre à ces critères
    \end{itemize}
  \end{customblock}

  \pause
  \begin{exampleblock}{Viabilité des modèles hybrides}
    \begin{itemize}  
    \item 1. Les représentations des réseaux sont pertinents avec des modèles frugaux
    \item 2. Le rapport qualité/temps de calculs peut être à notre avantage
    \item 3. La généralisation à des échantillons variés est possible
    \end{itemize}
  \end{exampleblock}
\end{frame}

\subsection{Perspectives}
\begin{frame}{\secname~- \subsecname}
  \begin{columns}
    \begin{column}{0.3\linewidth}
      \begin{figure}
        \centering
        \includegraphics[width=1\linewidth]{25.png}
      \end{figure}
    \end{column}
    \begin{column}{0.7\linewidth}
      \begin{block}{Perspective 1 – S'affranchir des gros réseaux}
        \begin{itemize}
        \item \alert{Limite}: Dépendance à \emph{VQGAN}
        \item Propriétés intéressantes: harmonisation, agrégation texturée

        \item \alert{Idée}: Apprendre des réseaux plus légers qui isolent ces propriétés
        \item Sans réseau de neurones ?
        \end{itemize}

      \end{block}
    \end{column}
  \end{columns}
\end{frame}

\begin{frame}{\secname~- \subsecname} % illustrer avec paysages
  \begin{columns}
    \begin{column}{\linewidth}
      \begin{block}{Perspective 2 – Étendre les capacités du modèle}
        \begin{itemize}
        \item \alert{Limite}: Génération de visages alignés seulement
        \item D'autres modalités non-triviales ?

        \item Tests sur auto-encodeurs génériques et visages non-recalés non concluants
        \item \alert{Idée}: Déléguer la tâche à un réseau léger
          (initialisation), puis raffiner avec des patchs
        \end{itemize}

      \end{block}
    \end{column}
  \end{columns}
\end{frame}


% the final page
\makethanks

\section{Suppléments}
\begin{frame}
  \begin{beamercolorbox}[sep=15pt,center,shadow=true,rounded=true]{title}
    \LARGE\bfseries \secname
  \end{beamercolorbox}
\end{frame}


\subsection{LatentPatch}
\begin{frame}{\secname~- \subsecname}
  \framesubtitle{Biais de performance}
  \begin{figure}
    \centering
    \includegraphics[width=0.8\linewidth]{19.pdf}
  \end{figure}
\end{frame}

\begin{frame}{\secname~- \subsecname}
  \framesubtitle{Comparaison quantitative - Precision}
  \begin{figure}
    \includegraphics[width=0.8\linewidth]{20.pdf}
    \caption{\small Calculé avec \emph{Improved Precision/Recall} [Kynkäänniemi et al., 2019]}
  \end{figure}
\end{frame}

\begin{frame}{\secname~- \subsecname}
  \framesubtitle{Comparaison quantitative - Rappel}
  \begin{figure}
    \includegraphics[width=0.8\linewidth]{21.pdf}
    \caption{\small Calculé avec \emph{Improved Precision/Recall} [Kynkäänniemi et al., 2019]}
  \end{figure}
\end{frame}
 
\begin{frame}{\secname~- \subsecname}
  \framesubtitle{Coût computationnel}
  \begin{figure}
    \includegraphics[width=0.8\linewidth]{22.png}
  \end{figure}
  \begin{itemize}
  \item En échange d'un coût en paramètres raisonnable, notre méthode
    \emph{LatentPatch} peut se passer d'entraînement pour des performances
    similaires à \emph{FastGAN}.
  \end{itemize}
\end{frame}

\subsection{GMM latent}
\begin{frame}{\secname~- \subsecname}
  \framesubtitle{Comparaisons qualitative}
  \begin{figure}
    \includegraphics[width=0.9\linewidth]{comp_tab}
  \end{figure}
  \scriptsize (1) [Esser et al., 2021] ; (2) [Liu et al., 2020]
\end{frame}

\end{document}
