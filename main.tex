\documentclass[aspectratio=169, 22pt]{beamer}
\makeatletter
\def\input@path{{./sty/}}
\makeatother

\usetheme{greyc}
%Choose your language
\def\lang{french} %set it to english or french to have the logo in the good language

\graphicspath{{./img/},{./figs/}}
\usepackage{amsmath}
\usepackage[justification=centering]{caption}


\title[Modèles hybrides pour la génération d'images]{Des modèles hybrides combinant représentations neuronales profondes et méthodes non-paramétriques à patchs pour la génération d'images photoréalistes}
\subtitle{}
\author[Benjamin Samuth]{Benjamin Samuth}
\institute[Normandie University]{Normandie Univ, UNICAEN, ENSICAEN, CNRS, GREYC, Caen, FRANCE}
\email{benjamin.samuth@unicaen.fr}
\web{}

\begin{document}

\begin{frame}
  \titlepage
\end{frame}

% ============================================
% ============================================
\section{Introduction}
\begin{frame}
  \begin{beamercolorbox}[sep=15pt,center,shadow=true,rounded=true]{title}
    \LARGE\bfseries \secname
  \end{beamercolorbox}
\end{frame}

\subsection{Contexte}
\begin{frame}{\secname~- \subsecname}
  \framesubtitle{Génération d'images~?}
  \begin{columns}
    \begin{column}{0.5\linewidth}      
      \begin{itemize}
      \item Créer de \alert{nouvelles} images
      \item Mais fidèles aux données de références
      \item Avec ou sans d'autres images d'entrée(s)
      \item Peuvent être conditionnées par un utilisateur
      \end{itemize}
    \end{column}
    \begin{column}{0.5\linewidth}
    \end{column}
  \end{columns}
\end{frame}

\begin{frame}{\secname~- \subsecname}
  \framesubtitle{La rapide évolution du domaine}
  \begin{figure}
    \centering
    \includegraphics[width=0.8\linewidth]{1.png}
    \caption{L'évolution du domaine de la génération d'images. La
      \alert{qualité}, mais aussi les \alert{possibilités sémantiques} ont nettement
      augmenté grâce aux techniques d'apprentissage profond. \alert{TODO: Sources}}
  \end{figure}
\end{frame}


\subsection{Enjeux et motivations}
\begin{frame}{\secname~- \subsecname}
  \framesubtitle{Enjeux}
  \begin{columns}
    \begin{column}{0.5\linewidth}
      Les modèles de générations de l'état de l'art sont dans la plupart des cas~:
      \begin{itemize}
      \item complexes,
      \item souvent inconsistents,
      \item volumineux en paramètres (contrainte matérielle),
      \item gourmands en données,
      \item longs et coûteux à entraîner.
      \end{itemize}
    \end{column}
    \begin{column}{0.5\linewidth}
      \begin{alertblock}{Des conséquences alarmantes}
      \begin{itemize}
      \item Désinformation (\emph{Deep Fakes})
      \item Propagande
      \item Violation de droit d'auteurs
      \item Sans sécurisation des données
      \item Coût écologique (énergie)
      \end{itemize}        
      \end{alertblock}
    \end{column}
  \end{columns}

  \vfill
  \begin{exampleblock}{}
    \centering
  À l'heure actuelle, les modèles de génération posent des
  \alert{problèmes} à la fois \alert{éthique} et \alert{de société}.
  \end{exampleblock}
\end{frame}

\begin{frame}{\secname~- \subsecname}
  \framesubtitle{Challenges}
  \begin{center}
    \textbf{Quels critères pour faire face à ces enjeux ?}
  \end{center}
  \begin{columns}
    \begin{column}{0.33\linewidth}
      \begin{block}{Modèles photoréalistes}
        \begin{itemize}
        \item \small Qualité d'images
        \item \small Fidélité aux données
        \item \small Génération variée
        \end{itemize}
      \end{block}
      \vfill
    \end{column}
    \begin{column}{0.33\linewidth}
      \begin{block}{Modèles légers}        
        \begin{itemize}
        \item \small Faible nombre de paramètres
        \item \small Faible volume de données
        \item \small Le moins d'opérations possible
        \end{itemize}
      \end{block}
      \vfill
    \end{column}
    \begin{column}{0.33\linewidth}
      \begin{block}{Modèles explicables}
        \begin{itemize}
        \item \small Maîtrise ds procédés
        \item \small Maîtrise des données
        \item \small Reproductibilité
        \item \small Interprétabilité
        \end{itemize}
      \end{block}      
    \end{column}
  \end{columns}
    
\end{frame}

\subsection{Méthodes ``hybrides''}
\begin{frame}{\secname~- \subsecname}
  \begin{columns}
    \begin{column}{0.5\linewidth}
      \begin{block}{Méthodes frugales}
        \begin{itemize}
        \item \small Légers
        \item \small Explicables
        \item \small Nécessitent peu de données
        \item \small Mais au champs d'application restreint
        \end{itemize}
      \end{block}      
    \end{column}
    
    \begin{column}{0.5\linewidth}
      \begin{block}{Modèles profonds}
        \begin{itemize}
        \item \small Génériques
        \item \small Photoréalistes
        \item \small À entraîner qu'une seule fois
        \item \small Mais peu explicables, souvent lourds, ...
        \end{itemize}
      \end{block}      
    \end{column}
  \end{columns}

  \vfill
  \begin{exampleblock}{\centering Des méthodes hybrides pour la génération d'images}
    \centering
    \alert{Idée}~: Combiner méthodes frugales et modèles profonds pour
    conserver les propriétés pertinentes de chacun
  \end{exampleblock}
\end{frame}

\subsection{Publications}
\begin{frame}{\secname~- \subsecname}
  \alert{TODO: Publications ICI}
\end{frame}


% ============================================
% ============================================
\section{Concepts préliminaires}
\begin{frame}
  \begin{beamercolorbox}[sep=15pt,center,shadow=true,rounded=true]{title}
    \LARGE\bfseries \secname
  \end{beamercolorbox}
\end{frame}

\subsection{Méthodes à patchs}
\begin{frame}{\secname~- \subsecname}
  \framesubtitle{Principe}
\end{frame}

\begin{frame}{\secname~- \subsecname}
  \framesubtitle{Propriété}
\end{frame}

\subsection{Auto-encodeurs}
\begin{frame}{\secname~- \subsecname}
  \framesubtitle{Principe}
\end{frame}

% ============================================
% ============================================
\section{Transfert de style par contrainte de patchs}
\begin{frame}
  \begin{beamercolorbox}[sep=15pt,center,shadow=true,rounded=true]{title}
    \LARGE\bfseries \secname
  \end{beamercolorbox}
\end{frame}

\subsection{Principe}
\begin{frame}{\secname~- \subsecname}
  \framesubtitle{Définition du transfert de style}
\end{frame}

\begin{frame}{\secname~- \subsecname}
  \framesubtitle{Modélisation du ``style artistique''}
\end{frame}

\subsection{Motivation}
\begin{frame}{\secname~- \subsecname}
  \framesubtitle{Pourquoi le transfert de style ?}
\end{frame}

\subsection{Méthode}
\begin{frame}{\secname~- \subsecname}
  \framesubtitle{Patchs plus proches voisins}
\end{frame}

\begin{frame}{\secname~- \subsecname}
  \framesubtitle{Multi-échelle}
\end{frame}

\begin{frame}{\secname~- \subsecname}
  \framesubtitle{Pénalisation d'occurrence}
\end{frame}

\subsection{Résultats}
\begin{frame}{\secname~- \subsecname}
\end{frame}

\begin{frame}{\secname~- \subsecname}
  \framesubtitle{Comparaisons}
\end{frame}



% ============================================
% ============================================
\section{Patchs latents pour la génération}
\begin{frame}
  \begin{beamercolorbox}[sep=15pt,center,shadow=true,rounded=true]{title}
    \LARGE\bfseries \secname
  \end{beamercolorbox}
\end{frame}

\subsection{Contexte}
\begin{frame}{\secname~- \subsecname}
  \framesubtitle{Génération de visages}
\end{frame}

\begin{frame}{\secname~- \subsecname}
  \framesubtitle{Génération \emph{``few-shot''}}
\end{frame}

\subsection{\emph{VQGAN}}
\begin{frame}{\secname~- \subsecname}
  \framesubtitle{[Esser et al., 2021]}
\end{frame}

\subsection{Méthode}
\begin{frame}{\secname~- \subsecname}
  \framesubtitle{Distance de patchs latents}
\end{frame}

\begin{frame}{\secname~- \subsecname}
  \framesubtitle{Génération séquentielle auto-régressive}
\end{frame}

\subsection{Expériences}
\begin{frame}{\secname~- \subsecname}
  \framesubtitle{Résultats}
\end{frame}

\begin{frame}{\secname~- \subsecname}
  \framesubtitle{Pertinence de l'auto-encodeur}
\end{frame}

\begin{frame}{\secname~- \subsecname}
  \framesubtitle{Explicable par design}
\end{frame}

\subsection{Évaluation}
\begin{frame}{\secname~- \subsecname}
  \framesubtitle{Comparaison qualitative}
\end{frame}

\begin{frame}{\secname~- \subsecname}
  \framesubtitle{Comparaison quantitative}
\end{frame}

\begin{frame}{\secname~- \subsecname}
  \framesubtitle{Coût computationnel}
\end{frame}

\subsection{Perspectives}
\begin{frame}{\secname~- \subsecname}
  \framesubtitle{Génération de texture}
\end{frame}

\begin{frame}{\secname~- \subsecname}
  \framesubtitle{Application aux paysages}
\end{frame}

\begin{frame}{\secname~- \subsecname}
  \framesubtitle{Limites à la généralisation aux grands volumes de données}
\end{frame}

% ============================================
% ============================================
\section{Modèle de mixture gaussiens latents}
\begin{frame}
  \begin{beamercolorbox}[sep=15pt,center,shadow=true,rounded=true]{title}
    \LARGE\bfseries \secname
  \end{beamercolorbox}
\end{frame}

\subsection{Méthode}
\begin{frame}{\secname~- \subsecname}
  \framesubtitle{Modèle latent de mixture de gaussiennes}
\end{frame}

\begin{frame}{\secname~- \subsecname}
  \framesubtitle{Échantillonnage à faible rang}
\end{frame}

\begin{frame}{\secname~- \subsecname}
  \framesubtitle{Raffinement par recherche de patchs plus proches voisins}
\end{frame}

\subsection{Résultats}
\begin{frame}{\secname~- \subsecname}
  \framesubtitle{Comparaisons}
\end{frame}
% ============================================
% ============================================
\section{Conclusion}
\begin{frame}
  \begin{beamercolorbox}[sep=15pt,center,shadow=true,rounded=true]{title}
    \LARGE\bfseries \secname
  \end{beamercolorbox}
\end{frame}

\subsection{Résumé}
\begin{frame}{\secname~- \subsecname}
\end{frame}

\subsection{Perspectives}
\begin{frame}{\secname~- \subsecname}
\end{frame}

% the final page
\makethanks

\section{Suppléments}

\end{document}
